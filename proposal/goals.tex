\section{Our Approach}

We plan use \textit{dataflow analysis} as implementation
framework for our analysis. 
The analysis shall warn the verification specialist
whenever it finds the usage of an unsanitized tainted value
at a sensible function call. The tool shall enable users
to define sensible program functions considered as program
sinks. The tool will assume the following functions as
program sources: 




\subsection{Taint Propagation}\label{analysis:propagation}

\begin{table}[h!]
\begin{center}
  \begin{tabular}{|l|c|c|}
  \hline
  Operations 				& 	Tainted Source Input	&	Target Tainted Input	\\ \hline
  $rStr = tStr$				&	$tStr$					&	$rStr$					\\ \hline
  $rStr = tStr + someStr$	& 	$tStr$					&	$rStr$					\\ \hline
  \end{tabular}
  \caption{Excerpt of Taint Propagation Rules}\label{analysis:propagation:t1}
\end{center}
\end{table}	

Taint propagation can be \textit{dataflow} or \textit{control-flow} based.
Dataflow based propagation are explicit and exist due to data
dependencies in the program. Statement $s_u$ is data dependent
on statement $s_d$ if there exists a \textit{def-use} relationship
between $s_d$ and $s_u$. Figure~\ref{fig:dfpropagation} illustrates
an example of dataflow taint propagation. 
%The statement at line $3$
%is data dependent on the statement at line $3$ because variable
%\texttt{\$credit} is defined at line $2$ and used at line $3$.

\begin{figure}[!h]
\begin{center}
\EmbedCode{c}{dfpropagation.c}{left}
\end{center}
\caption{Example code for dataflow taint propagation}
\label{fig:dfpropagation}
\end{figure}

Control-flow based propagation are implicit and are due to
control dependencies. A statement $s_t$ is control dependent on
a statement $s_s$ if $s_s$ contains a predicate $p$ and the execution
of $s_t$ may depend on $p$'s evaluation.
%Figure~\ref{fig:cfpropagation} illustrates an example of
%control-flow taint propagation: after invocation of the sanitizer function
%\myvar{ctype\_digit}, if the control flows in the else-branch, then uses of
%variable \myvar{amount} will not be tainted. On the other hand,
%\myvar{amount} remains tainted in the if-branch.

\begin{figure}[!h]
\begin{center}
\EmbedCode{c}{cfpropagation.c}{left}
\end{center}
\caption{Example code for control-flow taint propagation}
\label{fig:cfpropagation}
\end{figure}

Given a tainted input \texttt{tStr}, table~\ref{analysis:propagation:t1}
describes how program operations propagate taint information. The table
will be extended as the project advances.
We assume that statements involving arithmetic operators (e.g., $+, -$, etc.)
are transformed in a three-address code. The same applies for string
concatenations that use $+$.

\subsection{Sanitizers}\label{analysis:sanitizers}

Sanitizer functions, which change the state of tainted values
to untainted, will be defined by users.

\subsection{Pointer Analysis}\label{analysis:pointers}

\section{Implementation}

We are going to implement our analysis on top of the LLVM
compiler framework \cite{Lattner:2004:LLVM}.
We plan to use \cite{Lhotak:2011:PAE} as points-to analysis
framework. \cite{Lhotak:2011:PAE} refines Andersen's
flow-insensitive points-to analysis and adds flow-sensitivity
to it whenever possible.