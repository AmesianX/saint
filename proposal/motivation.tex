\section{Motivation}

Security vulnerabilities may incur severe lost to users.
EXAMPLE!!!

Taint analysis aims at finding program external inputs
that may be used without sanitizing checks.
We now introduce key terms used in taint analysis.
A \textit{source} is a program location that may produce
an untrusted value, i.e, a value from an untrusted program
input (e.g. return value from a system call, user input). 
Values from sources are qualified as \textit{tainted}.
A \textit{sanitizer} is a function that checks tainted
values. A tainted value becomes safe (trusted, untainted)
after it has been sanitized.
\textit{Sinks} are program locations that may use tainted
values (e.g. a function call may have a tainted value as
actual argument).

Taint analysis proceeds by first tagging values from sources
as tainted . Once tagged, Input data taint information is
propagated through the program.
\textit{Taint propagation} is the process of marking as tainted
values resulting from operations involving tainted data. In the
last step, taint analysis emits a warning whenever a tainted
value is used at a sink location.