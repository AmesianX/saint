\section{Motivation}

Software vulnerabilities are security threats
that exist in an application when users may exercise
unauthorized  control of the application through supplied input.
Such unauthorized control are classified in different
categories : \textit{buffer overflow}, \textit{SQL
injection}, \textit{cross site scripting}, etc.

Both dynamic and static techniques exist to alleviate
security vulnerabilities in software. 
This project aims at using \textit{\textbf{static taint analysis}}
to prevent software vulnerabilities.
Taint analysis searches program external inputs
usage that may allow users to gain unauthorized control
over the program. 

We now introduce key terms used in taint analysis.
A \textit{source} is a program location that may produce
an \textit{tainted value}, i.e, a value from an untrusted
program input (e.g. return value from a system call, user input). 
A \textit{sanitizer} is a function that validates tainted
values and make them safe (trusted, untainted) for further
program usage.
\textit{Sinks} are program locations that may use tainted
values (e.g. a function call may have a tainted value as
actual argument).

Taint analysis proceeds by first tagging values from sources
as tainted . Once tagged, Input data taint information is
propagated through the program.
\textit{Taint propagation} is the process of marking as tainted
values resulting from operations involving tainted data. In the
last step, taint analysis emits a warning whenever a tainted
value is used at a sink location.