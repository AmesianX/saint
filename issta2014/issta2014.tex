% This is "sig-alternate.tex" V2.0 May 2012
% This file should be compiled with V2.5 of "sig-alternate.cls" May 2012
%
% This example file demonstrates the use of the 'sig-alternate.cls'
% V2.5 LaTeX2e document class file. It is for those submitting
% articles to ACM Conference Proceedings WHO DO NOT WISH TO
% STRICTLY ADHERE TO THE SIGS (PUBS-BOARD-ENDORSED) STYLE.
% The 'sig-alternate.cls' file will produce a similar-looking,
% albeit, 'tighter' paper resulting in, invariably, fewer pages.
%
% ----------------------------------------------------------------------------------------------------------------
% This .tex file (and associated .cls V2.5) produces:
%       1) The Permission Statement
%       2) The Conference (location) Info information
%       3) The Copyright Line with ACM data
%       4) NO page numbers
%
% as against the acm_proc_article-sp.cls file which
% DOES NOT produce 1) thru' 3) above.
%
% Using 'sig-alternate.cls' you have control, however, from within
% the source .tex file, over both the CopyrightYear
% (defaulted to 200X) and the ACM Copyright Data
% (defaulted to X-XXXXX-XX-X/XX/XX).
% e.g.
% \CopyrightYear{2007} will cause 2007 to appear in the copyright line.
% \crdata{0-12345-67-8/90/12} will cause 0-12345-67-8/90/12 to appear in the copyright line.
%
% ---------------------------------------------------------------------------------------------------------------
% This .tex source is an example which *does* use
% the .bib file (from which the .bbl file % is produced).
% REMEMBER HOWEVER: After having produced the .bbl file,
% and prior to final submission, you *NEED* to 'insert'
% your .bbl file into your source .tex file so as to provide
% ONE 'self-contained' source file.
%
% ================= IF YOU HAVE QUESTIONS =======================
% Questions regarding the SIGS styles, SIGS policies and
% procedures, Conferences etc. should be sent to
% Adrienne Griscti (griscti@acm.org)
%
% Technical questions _only_ to
% Gerald Murray (murray@hq.acm.org)
% ===============================================================
%
% For tracking purposes - this is V2.0 - May 2012

\documentclass{sig-alternate}


\usepackage{amsmath}
\usepackage{verbatim}
%\usepackage[pdftex]{graphicx}
\usepackage{xcolor}
\usepackage{hyperref}
\hypersetup{
  filebordercolor=white,
  urlbordercolor=white %Remove boxes around link for url
  }
  
\usepackage{paralist}
\usepackage{amsmath}
\usepackage{amssymb}
\usepackage{xspace}
\usepackage{url}
\usepackage{listings}
\usepackage{color}
\usepackage[lined,resetcount]{algorithm2e}
\usepackage{proof}
\usepackage{relsize}
\usepackage{appendix}
\usepackage{tabularx}
\usepackage[justification=centering]{caption}

\newcommand{\EmbedCode}[3]{
	\lstset{language=#1, frame=none, backgroundcolor=\color{white},
			rulecolor=, xleftmargin=5.0ex}
	%\lstset{linewidth=\columnwidth}
	\lstset{commentstyle=\textit, stringstyle=\upshape,showspaces=false}
	\lstset{frame=none, showstringspaces=false, numbers=#3,
			numberblanklines=false}
	\lstinputlisting[]{#2} 
}

\newcommand{\set}[1]{\{#1\}}

\newcommand{\main}{\texttt{main}\xspace}
\newcommand{\even}{\texttt{even}\xspace}
\newcommand{\odd}{\texttt{odd}\xspace}
\newcommand{\compute}{\texttt{compute}\xspace}

\newcommand{\Aset}{\mathcal{A}\xspace}
\newcommand{\Pset}{\mathcal{T}\xspace}

\newcommand{\mathbold}[1]{\text{\textbf{#1}}}

\newcommand{\copydef}{\mathbold{COPY}\ [p = q]\xspace}
\newcommand{\loaddef}{\mathbold{LOAD}\ [p = *q]\xspace}
\newcommand{\addrofdef}{\mathbold{ADDROF}\ [p = \&a]\xspace}
\newcommand{\storedef}{\mathbold{STORE}\ [*p = q]\xspace}
\newcommand{\sourcedef}{\mathbold{SOURCE}\ [r = \text{call}\ \text{source}(f_0, f_1, ..., f_n)]\xspace}
\newcommand{\calldef}{\mathbold{CALL}\ [r = \text{call}\ \text{func}(a_0, a_1, ..., a_n)]}
\newcommand{\callinsdef}{\mathbold{CALL-INS}\ [r = \text{call}\ \text{func}(a_0, a_1, ..., a_n)]}
\newcommand{\sinkdef}{\mathbold{SINK}\ [r = \text{call}\ \text{sink}]}

\newcommand{\ALLOC}{\text{\textbf{[ALLOC]}}\xspace}
\newcommand{\COPY}{\text{\textbf{[COPY]}}\xspace}
\newcommand{\LOAD}{\text{\textbf{[LOAD]}}\xspace}
\newcommand{\ADDROF}{\text{\textbf{[ADDROF]}}\xspace}
\newcommand{\STORE}{\text{\textbf{[STORE]}}\xspace}
\newcommand{\CALL}{\text{\textbf{[CALL]}}\xspace}
\newcommand{\SOURCE}{\text{\textbf{[SOURCE]}}\xspace}

\newcommand{\alloct}{\texttt{ALLOC}\xspace}
\newcommand{\copyt}{\texttt{COPY}\xspace}
\newcommand{\loadt}{\texttt{LOAD}\xspace}
\newcommand{\storet}{\texttt{STORE}\xspace}
\newcommand{\callt}{\texttt{CALL}\xspace}

\newcommand{\DSA}{\texttt{DSA}\xspace}
\newcommand{\waint}{\textsc{Waint}\xspace}

\begin{document}
%
% --- Author Metadata here ---
\conferenceinfo{WOODSTOCK}{'97 El Paso, Texas USA}
%\CopyrightYear{2007} % Allows default copyright year (20XX) to be over-ridden - IF NEED BE.
%\crdata{0-12345-67-8/90/01}  % Allows default copyright data (0-89791-88-6/97/05) to be over-ridden - IF NEED BE.
% --- End of Author Metadata ---

\title{{\ttlit Waint}, Extensible Hybrid Analysis
Tool for Bug Finding and Vulnerability Detection\titlenote{(Produces the permission block, and
copyright information). For use with
SIG-ALTERNATE.CLS. Supported by ACM.}}
\subtitle{[Extended Abstract]
\titlenote{A full version of this paper is available as
\textit{Author's Guide to Preparing ACM SIG Proceedings Using
\LaTeX$2_\epsilon$\ and BibTeX} at
\texttt{www.acm.org/eaddress.htm}}}
%
% You need the command \numberofauthors to handle the 'placement
% and alignment' of the authors beneath the title.
%
% For aesthetic reasons, we recommend 'three authors at a time'
% i.e. three 'name/affiliation blocks' be placed beneath the title.
%
% NOTE: You are NOT restricted in how many 'rows' of
% "name/affiliations" may appear. We just ask that you restrict
% the number of 'columns' to three.
%
% Because of the available 'opening page real-estate'
% we ask you to refrain from putting more than six authors
% (two rows with three columns) beneath the article title.
% More than six makes the first-page appear very cluttered indeed.
%
% Use the \alignauthor commands to handle the names
% and affiliations for an 'aesthetic maximum' of six authors.
% Add names, affiliations, addresses for
% the seventh etc. author(s) as the argument for the
% \additionalauthors command.
% These 'additional authors' will be output/set for you
% without further effort on your part as the last section in
% the body of your article BEFORE References or any Appendices.

\numberofauthors{3} %  in this sample file, there are a *total*
% of EIGHT authors. SIX appear on the 'first-page' (for formatting
% reasons) and the remaining two appear in the \additionalauthors section.
%
\author{
% You can go ahead and credit any number of authors here,
% e.g. one 'row of three' or two rows (consisting of one row of three
% and a second row of one, two or three).
%
% The command \alignauthor (no curly braces needed) should
% precede each author name, affiliation/snail-mail address and
% e-mail address. Additionally, tag each line of
% affiliation/address with \affaddr, and tag the
% e-mail address with \email.
%
% 1st. author
\alignauthor
Xavier N. Noumbissi\\
% 2nd. author
\alignauthor
Vijay Ganesh\\
% 3rd. author
\alignauthor
Patrick Lam\\
\and  % use '\and' if you need 'another row' of author names
       \affaddr{Department of Electrical and Computer Engineering}\\
       \affaddr{University of Waterloo}\\
       \affaddr{\{xnoumbis,vganesh,plam\}@uwaterloo.ca}\\
}
% There's nothing stopping you putting the seventh, eighth, etc.
% author on the opening page (as the 'third row') but we ask,
% for aesthetic reasons that you place these 'additional authors'
% in the \additional authors block, viz.
%\additionalauthors{Additional authors: John Smith (The Th{\o}rv{\"a}ld Group,
%email: {\texttt{jsmith@affiliation.org}}) and Julius P.~Kumquat
%(The Kumquat Consortium, email: {\texttt{jpkumquat@consortium.net}}).}
\date{24 January 2014}
% Just remember to make sure that the TOTAL number of authors
% is the number that will appear on the first page PLUS the
% number that will appear in the \additionalauthors section.

\maketitle
\begin{abstract}
Businesses increasingly use software. This is
even more relevant for companies relying on e-commerce. However,
software is error-prone and contain several bugs. Security
bugs are one of the major problems faced by
companies today. In the worst case, security bugs
enable unauthorized users to gain full control of an application.
This paper addresses the issue of finding security vulnerabilities
in applications through static taint analysis.
We present an interprocedural taint analysis for C programs,
implemented in the LLVM framework.
Our analysis comes in two flavors: a context-insensitive and
a context-sensitive version that may complement each other.
\end{abstract}

% A category with the (minimum) three required fields
\category{H.4}{Information Systems Applications}{Miscellaneous}
%A category including the fourth, optional field follows...
\category{D.2.8}{Software Engineering}{Metrics}[complexity measures, performance measures]

\terms{Theory}

\keywords{ACM proceedings, \LaTeX, text tagging}

\section{Introduction}
Software vulnerabilities are security threats
that exist in an application when users may exercise
unauthorized  control of the application through supplied input.
Such unauthorized control are classified in different
categories depending on how the user gains control of
the application: buffer overflow, format string attacks,
SQL injection, cross-site scripting, etc.
Researchers have worked both on dynamic \cite{Clause:2007:Dytan,
Kiezun:2009:Ardilla}, static\cite{Tripp:2009:TAJ, Kiezun:2009:Ardilla,
Chang:2009:ICS, Parfait:2008, Jovanovic:2006:Pixy, livshits05finding,
Avots:2005:ISS, Shankar:2001:DFS}, and hybrid
techniques\cite{Trip:2011:HAJSA} to find  security
vulnerabilities in software. 

This paper presents a static taint analysis to detect software
vulnerabilities in C programs. The analysis is implemented
using the LLVM framework\cite{Lattner:2004:LLVM}, and does
not require any developer annotations. Our analysis
is interprocedural, and developers can choose between a
context-insensitive and a context-sensitive version. 

Taint analysis is also know as \textit{user-input dependence analysis}.
A taint analysis searches for program external inputs usage that
may allow users to gain unauthorized control over the program. 
In taint analysis, a \textit{source} is a program location
or statement that may produce a \textit{tainted value}, i.e,
a value from an untrusted program input (e.g. return value from a
system call, user input). 
A \textit{sink} is program location or statement that may use
a tainted value (e.g. a function call may have a tainted value as
actual argument), and thereby granting unauthorized control
to a malevolent user.
A \textit{sanitizer} is a function that validates tainted
values and make them safe (trusted, untainted) for further
program usage.

Taint analysis proceeds by first tagging values from sources
as tainted. Once tagged, the tainted values are propagated
through the program.
\textit{Taint propagation} is the process of marking as tainted, 
values that result from operations involving tainted data. This
can be an arithmetic operation, an assignment, etc. In the
last step, taint analysis emits a warning whenever a tainted
value is used at a sink location.

Taint propagation can be \textit{data flow} or \textit{control flow}
based. Data flow based taint propagation is explicit and exists
due to data dependencies in the program (e.g. assigning the value
of tainted variable $s_u$ to variable $s_d$).
Control flow based taint propagation is implicit and is due to
control dependencies (e.g. if tainted variable $s_t$ is used in a branch
condition, variables assigned in the condition statement branches
become tainted because their values have a dependency on the
tainted variable $s_t$).

This paper is organized as follows: Section~\ref{sec:example}
introduces our running example. Section~\ref{sec:llvm} gives
a brief overview of the LLVM framework's intermediate
representation. We present our analysis and an experimental
evaluation in Section~\ref{sec:analysis} and Section~\ref{sec:evaluation}
respectively. Finally, we discuss related work in Section~\ref{sec:related}
and conclude in Section~\ref{sec:conclusion}.

\section{Motivating Example}\label{sec:example} 

\begin{figure}
\begin{center}
\EmbedCode{c}{main.c}{left}
\end{center}
\caption{Motivating Example}
\label{fig:sample}
\end{figure}

Figure~\ref{fig:sample} shows $4$ C functions:
\main, \even, \odd, and \compute.

In \main{}, function \texttt{scanf} from the standard input/ouput
library gets an integer input from user at line $3$
and stores it in variable \texttt{x}. \texttt{x} 
becomes \textit{tainted} since it holds a value from
the environment which has not been validated and
sanitized.
\texttt{x} is later as argument to \even{} and \odd{} at
lines $4$ and $5$ respectively.

In \compute{}, variable \texttt{sum} gets tainted at
line $13$ through function \texttt{scanf}. Parameter
\texttt{x} is tainted only if it was passed tainted
at calling sites.
This is for instance the case in \main{} at line $6$.
Observe that tainted parameter \texttt{x} is used in
the conditional expression at line $12$, which constitute
a case of \textit{control-flow based taint propagation}.
Our analysis will not handle control-flow based taint
propagation\footnote{Detection of some of denial-of-service
vulnerabilities requires handling of control-flow based
taint propagation \cite{Chang:2009:ICS}}. 

Functions \even{} and \odd{} have one formal parameter
\texttt{x}. Both do not taint \texttt{x} in their code.
Thus \texttt{x} may be tainted in their body iff it was
passed as tainted at calling sites. This is the case
for \even{} at line $4$ in \main{}. Thus implying a tainted
parameter for \odd{} at line $32$ in \even{}.

\section{LLVM's Intermediate Representation}\label{sec:llvm}

This section gives an overview of LLVM's intermediate
representation (IR), which we use as basis for the
description of our analysis.

LLVM's IR uses partial static single assignment (SSA)
and assumes the existence of only two types of variables
in C code: \textit{top-level} and \textit{address-taken}
variables.

Top-level variables are in SSA form and can not be
accessed indirectly via a pointer. That is, their memory
address is never copied to another variable (i.e. they are
never applied the address-of operator). Top-level variables
are accessed with \alloct and \copyt instructions.
This paper denotes the set of top-level variables
with $\Pset$. For instance, $b1$, and $b2$ in Figure~\ref{fig:sample}
are top-level variables ($\{b1, b2\} \in \Pset$).

Address-taken variables are never accessed directly through
their first declared name. Address-taken variables are only
accessed indirectly through pointer variables with \loadt and
\storet instructions. In fact, address-taken variables
are those ones on which the address-of operator (\texttt{\&})
was applied. This paper uses $\Aset$ for the set of all address-taken
variables. Variable \texttt{x} in Figure~\ref{fig:sample} is
for instance an address-taken variable ($x \in \Aset$).

This paper describes the analysis by using the following abstract
instructions:
\begin{itemize}
\item $\copydef$: copy instruction. $p, q \in \Aset$.
\item $\loaddef$: load instruction. $q \in \Aset$.
\item $\storedef$: store instruction. $p \in \Aset, q \in \Pset$.
\item $\calldef$: call to a function. $r \in \Pset$
\item $\sourcedef$: call to a taint source function. $r \in \Pset$
%\item $\sinkdef$: call to a sink function. $r \in \Pset$
\end{itemize}


\section{Staged Taint Analysis}\label{sec:analysis}

\newcommand{\varset}{\mathit{Var}\xspace}
\newcommand{\instset}{\mathit{Inst}\xspace}
\newcommand{\procset}{\mathit{Proc}\xspace}
\newcommand{\formalsset}{\mathit{formals}\xspace}
\newcommand{\aliases}{\mathit{aliases}\xspace}
\newcommand{\intset}{\mathbb{N}^+}
\newcommand{\firstfunc}{\mathit{first}}
\newcommand{\toplevelfunc}{\mathit{toplevel}}
\newcommand{\taintfunc}{\mathit{taint}}
\newcommand{\pointsto}[2]{{pt}_{[#1]}(#2)}
\newcommand{\pointstobefore}[2]{{pt}_{[\overline{#1}]	}(#2)}
\newcommand{\pointstoafter}[2]{{pt}_{[\underline{#1}]}(#2)}

\newcommand{\myinflow}{\mathit{IN}}
\newcommand{\myoutflow}{\mathit{OUT}}
\newcommand{\ifff}{\mathit{iff}}
\newcommand{\aand}{\mathit{and}}
\newcommand{\mybigcup}[2]{\mathlarger{\bigcup_{#1}^{#2}}}

The taint analysis is interprocedural and runs
either context-insensitively or context-sensitively.
Any of the interprocedural analysis is always preceded
by a unique intraprocedural analysis that computes initial
information that are reused by the interprocedural analyses.
The intraprocedural analysis detects taint sources and
initializes a summary table which contains taint information
about program procedure's formal parameters and return value.
The use of a summary table allows fast access to key
information about program procedures. This is especially
useful during the subsequent interprocedural phases.
For instance, the intraprocedural analysis would
detect that variable \texttt{sum} of procedure \compute{} in
Figure~\ref{fig:sample}, which also holds the return
value of \compute{}, may be tainted due to the call
to \texttt{scanf} at line $13$. This information is then
reused by the context-insensitive analysis to infer that
variable y may hold a tainted value.
%Appendix~\ref{app:algo} presents pseudo code for all
%our analysis algorithms.

\paragraph{Taint Sources and Sinks}
Functions that taint variables (\textit{taint sources}) are discovered
during the intraprocedural analysis described later in this section.
The analysis considers a subset of standard C funtions to be taint sources:
{\tt getc, scanf, gets, fopen}, etc.
Developers are also allow to specify additional taint sources
by declaring them in a configuration file.

Taint sinks (functions that use tainted variables) are gradually
discovered during the various phases of the analysis. Developers
can specify sink functions by declaring them in a configuration
file. The analysis uses by default a subset of C functions as
sinks ({\tt  }, etc).

\paragraph{Taint Propagation}
The analysis only performs \textit{explicit taint propagation}
(data-flow taint propagation). Explicit taint propagation tracks variables
that are tainted due to assignment statements. The assignment in line $6$
of Figure~\ref{fig:sample} is an instance of explicit taint propagation.
Variable $y$ becomes tainted since it gets assigned the return value of
\texttt{compute}, which is tainted value.\\
The analysis does not perform \textit{implicit taint propagation}
(control-flow propagation), which is another type of taint propagation.
Implicit taint propagation also takes into account tainted variables
used in control conditions. For instance, if a tainted variable $x$ is
used in the boolean condition of an if-statement, then variables
assigned in both branches of the statement becomes tainted because
their value depend on tainted variable $x$.

\paragraph{Formalisms}
The analysis dataflow set is $\varset \times 2^\instset$,
where $\varset$ is the set of all program variables and
$\instset$ the set of all program statements. 
At a statement labelled $s$, the incoming dataflow set $\myinflow[s]$
is the set of program variables that are tainted before statement $s$.
If a variable $v$ is not tainted before statement $s$, then
$v \notin \myinflow[s]$; otherwise $\myinflow[s]$ contains $v$.

This paper uses the following elements to describe the
taint analysis:
$\varset$ is the set of all program variables,
$\procset$ is the set of all program functions and procedures\footnote{We
will use the terms function and procedure interchangeably in
the remainder of this paper.},
$\instset$ is the set of all program statements,
$\firstfunc{}: \procset \rightarrow \instset$ returns the first
statement of a function,
$\formalsset{}: \procset \rightarrow 2^{\varset}$ returns the
set of formal parameters of a function, 
$\toplevelfunc{}: \Aset \rightarrow \Pset$ returns the top level
variable of an address-taken variable,
$\pointstobefore{s}{}: \varset \rightarrow 2^\Aset$ returns the
points-to set of a variable before the statement labelled s, 
$\pointstoafter{s}{}: \varset \rightarrow 2^\Aset$ returns the
points-to set of a variable after the statement labelled s,
$\taintfunc: \intset \rightarrow bool$ returns \texttt{true}
if function $f$ always taints its $k^{th}$ formal parameter, and
$\aliases: \varset \rightarrow 2^\varset$ returns the alias set of
a variable as returned by the pointer analysis.

\paragraph{Pointer Analysis}
This paper's taint analysis is designed to work using the results of
a previously computed pointer analysis.
The current implementation uses the pointer analysis \DSA (\texttt{Data Structure Analysis}) \cite{DSA:PLDI07}.
\DSA is a field- and context-sensitive pointer analysis.
\DSA uses full heap cloning (by acyclic call paths) and 
scales well with programs in a size range of 100K-200K lines of C code. 
Algorithm~\ref{fig:insertTaint} illustrates how the taint analysis uses
the results of the pointer analysis using function \texttt{aliases}
to update data flowsets.
\begin{algorithm}
\caption{flowInsert. Insertion of Taint Information}\label{fig:insertTaint}
\LinesNumbered
\DontPrintSemicolon
\SetKwData{Caller}{caller}
\SetKwData{stmt}{s}
\SetKwData{Input}{IN}
\SetKwData{ValueV}{v}
\SetKwData{ValueW}{$a_v$}
\SetKwData{Output}{OUT}
\SetKwFunction{Aliases}{aliases}
\SetKwFunction{Func}{func}
\SetKwInOut{InData}{input}
\SetKwInOut{OutData}{output}

\InData{$\stmt: \mathit{Inst}, \ValueV: \mathit{Var}$}
\OutData{}
$\Output[\stmt] \leftarrow \Output[\stmt] \cup \{ \ValueV \}$\;
	\ForEach{$\ValueW \in \Aliases(\ValueV)$}{			
		$\Output[\stmt] \leftarrow \Output[\stmt] \cup \{ \ValueW \}$\;
	}
\end{algorithm}

\paragraph{Intraprocedural Analysis}
The intraprocedural analysis always runs first, before any of
the interprocedural analyses, and is responsible for discovering
initial taint sources. During the intraprocedural analysis,
program functions are analyzed in the reverse topological order
of the call graph (i.e starting from the leaves of the callgraph).
The analysis works on each function body and do not take into account
the interprocedural control flow. The computed data flow sets are
reused later by the subsequent interprocedural analyses. In particular,
taint information about function formal parameters and return
value is kept in a global summary table. All tainted variables due
to source functions are found during the intraprocedural analysis.
For instance, the intraprocedural analysis detects that variable
\texttt{sum} of function \compute{} in Figure~\ref{fig:sample}
may be tainted at line $13$ due to the call to \texttt{scanf}, which
the analysis considers as a taint source. 
Algorithm~\ref{fig:intraFlow} shows the flow function for
\callt statements, which handles the discovery of taint sources
during the intraprocedural analysis pass.
Flow functions for all other statement types are same as the ones
of Algorithm~\ref{fig:csInterFlow} in Appendix~\ref{app:algo}.
%%%%%%%%%%%%%%%%%%%%%%%%%%%%%%%%%%%%%%%%%%%%%%%%%%%%%%%%%%%%%%%%%%%%%%%%%%%%%%%%%%%%%%%%%%%%
\begin{algorithm}
\caption{Intraprocedural Analysis Flow Function for \callt statements}\label{fig:intraFlow}
%\SetAlgoNoLine
\LinesNumbered
\DontPrintSemicolon
\SetKwData{Caller}{caller}
\SetKwData{stmt}{s}
\SetKwData{VarA}{a}
\SetKwData{stmtF}{f}
\SetKwData{stmtR}{r}
\SetKwData{VarK}{k}
\SetKwData{summaryTable}{sumTable}
\SetKwData{Input}{IN}
\SetKwData{Output}{OUT}
\SetKwFunction{LSOURCE}{SOURCE}
\SetKwFunction{Type}{TypeOf}
\SetKwFunction{Taint}{taint}
\SetKwFunction{Source}{source}
\SetKwFunction{Formals}{formals}
\SetKwFunction{FlowInsert}{flowInsert}
\SetKwInOut{InData}{input}
\SetKwInOut{OutData}{output}

\InData{$\Caller: \mathit{Proc}, \stmt: \mathit{Inst}$}
\OutData{}
\Switch{$\Type(\stmt)$}{	
	\Case{$\LSOURCE\ [\stmtR = \mathit{call}\ \Source(\VarA_0, \VarA_1, ..., \VarA_n)]$}{
		\ForEach{$\VarK \in \set{0, 1, ..., n}$}{
			\If{$\Taint(\VarK)$}{			
				$\FlowInsert(\stmt, \VarA_k)$\;
			}			
		}
	}			
}
\ForEach{$\stmtF_k \in \Formals(\Caller)$}{
	\If{$\Output[\stmt] \neq \Input[\stmt]$ {\bf and} $\stmtF_k \in \Output[\stmt]$}{
		$\summaryTable[\Caller][\VarK] = 1$\;
	}
}
\end{algorithm}

\paragraph{Context-Insensitive Analysis}
The context-insensitive analysis starts from the \textit{main function}
(entry point of the program) and performs in the topological order
of the call graph. It uses function return value taint assumptions
from the intraprocedural analysis and applies them whenever available.
The context-insensitive analysis does not propagate taint information
from the caller to the callee that may happen through callee arguments.
It only uses function return taint information from the intraprocedural
analysis  and further propagates it.
For instance, the intraprocedural analysis of function \main{}
marks the return value of \compute (stored in variable \texttt{sum})
as tainted in the summary table. At line $6$ in \main, the context-insensitive
analysis  would take into account that the return value of \compute is
stored into variable \texttt{y}. Thus, \texttt{y} becomes a
tainted variable during the context-insensitive analysis.
Algorithm~\ref{fig:interFlow} in the following
illustrates the context-insensitive flow function for \callt statements.
%%%%%%%%%%%%%%%%%%%%%%%%%%%%%%%%%%%%%%%%%%%%%%%%%%%%%%%%%%%%%%%%%%%%%%%%%%%%%%%%%%%%%%%%%%%%
\begin{algorithm}
\caption{Context-Insentive Interprocedural Flow Function for
\callt statements}\label{fig:interFlow}
%\SetAlgoNoLine
%\SetAlgoLined
\LinesNumbered
\DontPrintSemicolon
\SetKwData{Caller}{caller}
\SetKwData{stmt}{s}
\SetKwData{ValueV}{v}
\SetKwData{VarA}{a}
\SetKwData{stmtF}{f}
\SetKwData{VarK}{k}
\SetKwData{ValueR}{r}
\SetKwData{summaryTable}{sumTable}
\SetKwData{Input}{IN}
\SetKwData{Output}{OUT}
\SetKwFunction{LCALL}{CALL}
\SetKwFunction{Func}{G}
\SetKwFunction{FlowInsert}{flowInsert}
\SetKwInOut{InData}{input}
\SetKwInOut{OutData}{output}

\InData{$\Caller: \mathit{Proc}, \stmt: \mathit{Inst}$}
\OutData{}
\Switch{$\Type(\stmt)$}{		
	\Case{$\LCALL\ [\ValueR = \mathit{call}\ \Func(a_0, a_1, ..., a_n)]$}{
		\If{$\summaryTable[\Func][n+1]$}{
			$\FlowInsert(\stmt, \ValueR)$\;						
		}			
		\ForEach{$\VarK \in \set{0, 1, ..., n}$}{
			\If{$\summaryTable[\Func][\VarK]$}{
				$\FlowInsert(\stmt, \VarA_k)$\;						
			}							
		}
	}				
}
\end{algorithm}

\paragraph{Context-Sensitive Analysis}
The context-sensitive analysis works in the topological order
of the call graph, and uses information from the summary table
that were produced by previous analyses. Even if the context-insensitive
analysis was run before, the context-sensitive analysis eventually
finds more precise information in the summary table. At call sites,
the context-sensitive analysis propagates actual parameters
taint information from the caller into the callee. At callees' exits,
newly computed taint information are propagated back from the
callee to the caller context. Algorithm~\ref{fig:contextIF}
illustrates the flow function for \callt statements during the
context-sensitive analysis. On the other hand, algorithm~\ref{fig:csInterFlow}
in Appendix~\ref{app:algo} presents the full context-sensitive
algorithm.
%%%%%%%%%%%%%%%%%%%%%%%%%%%%%%%%%%%%%%%%%%%%%%%%%%%%%%%%%%%%%%%%%%%%%%%%%%%%%%%%%%%%%%%%%%%%
\begin{algorithm}
\caption{csInterFlow: Context-Sentive Flow Function for \callt
statements}\label{fig:contextIF}
%\SetAlgoNoLine
\SetAlgoLined
\LinesNumbered
\DontPrintSemicolon
\SetKwData{Caller}{caller}
\SetKwData{stmt}{s}
\SetKwData{stmtA}{a}
\SetKwData{stmtF}{f}
\SetKwData{stmtR}{r}
\SetKwData{False}{false}
\SetKwData{Input}{IN}
\SetKwData{Output}{OUT}
\SetKwData{summaryTable}{sumTable}
\SetKwFunction{LCALL}{CALL}
\SetKwFunction{Type}{TypeOf}
\SetKwFunction{InterFlow}{csInterFlow}
\SetKwFunction{Func}{G}
\SetKwFunction{Formals}{formals}
\SetKwFunction{FlowInsert}{flowInsert}
\SetKwInOut{InData}{input}
\SetKwInOut{OutData}{output}

\InData{$\Caller: \mathit{Proc}, \stmt: \mathit{Inst}$}
\OutData{}
\Switch{$\Type(\stmt)$}{
%Context-sensitive		
	\Case{$\LCALL\ [\stmtR = \mathit{call}\ \Func(\stmtA_0, \stmtA_1, ..., \stmtA_n)]$}{
		\If{$\Caller \neq \Func$}{
			\If{$0 \in \summaryTable[\Func]$}{
				\ForEach{$\stmtF_k \in \Formals(\Func)$}{
					\If{$\summaryTable[\Func][k]$}{	
							$\FlowInsert(\stmt, \stmtA_k)$;											
					}
				}											
				$\InterFlow(\Caller, \Func)$\;			
				\ForEach{$\stmtF_k \in \Formals(\Func)$}{
					\If{$!\summaryTable[\Func][k] \ \aand\ \Output[\stmtF_k] \neq \Input[\stmtF_k]$}{		
						$\summaryTable[\Func][k] = 1$\;						
						$\FlowInsert(\stmt, \stmtA_k)$\;				
					}
				}					
			}
		}
	}				
}
\end{algorithm}

\paragraph{Recursive and Mutual Recursive Function Calls} 
The context-sensitive analysis computes results for simple recursive
functions only for their first calling context.
The analysis identifies mutual recursive function calls by finding
strongly connected components (SCC) in the call graph generated
by LLVM.

There is no particular treatment of mutual recursive function
calls during the context-insensitive analysis. Only the
procedure information from the summary table is used.

During the context-sensitive analysis, functions in the call
graph that builds a SCC are analyzed together.
That is, when a call to one function of a SCC happens,
the analysis starts from that function and propagates taint
information at call sites of all other functions in the SCC
set, but each function is analyzed once.

\paragraph{Complex Data Structures}

Given the limited scope of this project, we are not handling
complex data structures such as array and structures.
However, we plan to report the usage of tainted C structures
(\texttt{struct}) and arrays in future work.

\section{Experimental Evaluation}\label{sec:evaluation}

In the C programming language, \textit{format string vulnerabilities}
happen when an incorrect number of arguments is used when
calling variable-length functions. For instance, a call
such as
\begin{center}
{\tt printf(buf);} // format string vulnerability
\end{center}
may cause the program to crash because the function \texttt{printf}
expects its first parameter to be a format string. For
instance {\tt printf("\%s", buf)} would be a correct call.

We evaluated our taint analysis by attempting to detect
format string vulnerabilities of \texttt{printf}-like functions
in four different open-source C programs.
For this evaluation, the taint analysis emits a warning
whenever a format string vulnerability is encountered,
or whenever a tainted variable is used in a sink function
(e.g \texttt{snprintf}).

We ran the analysis on an \texttt{Intel i7 @ 3.4GHz} with
$4$ cores and $16$ GB of RAM, running {\tt Debian Wheezy}.
Table~\ref{tab:results} shows the analysis results of the
taint analysis, ran in the following order: intraprocedural
analysis, context-sensitive analysis, and context-insensitive
analysis.
\begin{table}[!hbtp]
\centering
\begin{tabular}{|l|c|c|c|c|}
\hline
{\bf Program}				&	{\bf Description}	&	{\bf SLOC}	& {\bf Warnings}	&	{\bf Time}	\\ \hline
{\tt mongoose}	($4.1$)		&	Web server			&	$4,339$		&	$173$			&	$8m23s$ 	\\ %\hline
{\tt vlc-input}	($4.1$)		&	Media player		&	$16,960$	&	$1$				&	$8m23s$ 	\\ %\hline
{\tt claws}		($3.9.3$)	&	Email client		& 	$142,890$ 	&	$5$				&	$4m19s$		\\ %\hline
{\tt apache}	($2.4.7$)	&	Web server			&	$144,356$	&	N.A				&	N.A	 		\\ \hline
\end{tabular}\caption{Taint Analysis Evaluation Results}\label{tab:results}
\end{table}

\paragraph{Discussion}

Our implementation used for this evaluation implements taint
propagation only for basic C language elements:
\begin{itemize}
  \item Assignment operations
  \item Copy and Store of pointers and scalar variables
\end{itemize} 
In future work, we plan to add taint propagation for other
C language elements such as {\tt casts, bit operations}.

We could not get the taint analysis running on the \texttt{apache}
web server because of a crash happening during initialization of
the \texttt{DSA} points-to analysis.


\section{Related Work}\label{sec:related}

\paragraph{\cite{Shankar:2001:DFS}} presents a tool to statically
detect format string vulnerabilities in C programs. The analysis
is based on \textit{type qualifiers} and type inference. Given a C program,
an initial subset of the program is annotated with the type qualifiers
\texttt{tainted} and \texttt{untainted}, which are new qualifiers
introduced by the authors. The paper defines a set
of inference rules for C language elements. The inference rules are used,
together with the initial program annotation to generate type qualifier
constraints over the program. The analysis then warns the user
about a format string vulnerability whenever the program does not
type checks due to an unsatisfiable constraint. The presented tool
is built on top of the type qualifier framework \textit{CQUAL} 
\cite{Foster:pldi99}.

\paragraph{\cite{Dimitru:2009:STAC}} presents STAC: a static taint
analysis for C, implemented as a type system. The paper defines a type
domain consisting of the values \texttt{Tainted} and \texttt{Untainted}.
A type environment is a mapping from variables to type domain values.
STAC defines a set of inference rules that allows to compute
type environments for C program statements and functions. The paper
also proves the soundness of the presented taint analysis.

\paragraph{\cite{Jovanovic:2006:Pixy}} presents \texttt{Pixy}: a tool
that statically scans for cross site scripting vulnerabilities in PHP
scripts. Pixy and and our project have similar goals. The main
difference is that Pixy performs a static analysis whereas we plan
to develop a dynamic analysis. Pixy implements a flow-sensitive,
interprocedural and context-sensitive dataflow analysis, which
is based on an alias and literal analysis. After evaluating
their tool on three PHP applications, they authors observed 
a false positives rate of around $50\%$.

%Dynamic taint analysis
\paragraph{\cite{Clause:2007:Dytan}} presents \texttt{Dytan}: a generic
framework for implementing dynamic taint analyses. Dytan generates
dynamic analyses that perform on x86 binaries. The source code of the
analyzed application does not need to be available.
Developers specify a dynamic taint analysis with Dytan by giving
(1) taint sources, (2) taint sinks, (3) and a propagation policy.
Taint sources can be variable names, function-return values, and
data read from I/O stream such as a file or a network connection.
Taint sinks are specified by memory/code location or based on the
usage scenario. Memory and code location specifications may include 
a variable's name and scope, a procedure name and the index of a formal parameter, 
a position in the source code, an offset from the base address of the program,
and the entry or exit points of a procedure.
A usage scenario specification takes into account whether the developer
wants check a taint information before the execution instructions of a given type
(e.g., a jump instruction). For that, the developer needs to
specify an instruction type and its corresponding checking operation.
Dytan provides dataflow and control-flow based propagation policies.
Dataflow propagation policies are based on (explicit) data dependencies.
Control-flow dependencies relies on (implicit) control dependencies. 

%Taint Analysis for managed languages

\section{Conclusion}\label{sec:conclusion}

Integration in IDE

%ACKNOWLEDGMENTS are optional
\section{Acknowledgments}
This section is optional; it is a location for you
to acknowledge grants, funding, editing assistance and
what have you.  In the present case, for example, the
authors would like to thank Gerald Murray of ACM for
his help in codifying this \textit{Author's Guide}
and the \textbf{.cls} and \textbf{.tex} files that it describes.

%
% The following two commands are all you need in the
% initial runs of your .tex file to
% produce the bibliography for the citations in your paper.
\bibliographystyle{abbrv}
\bibliography{../final-report/waint}  % sigproc.bib is the name of the Bibliography in this case
% You must have a proper ".bib" file
%  and remember to run:
% latex bibtex latex latex
% to resolve all references
%
% ACM needs 'a single self-contained file'!
%
%APPENDICES are optional
%\balancecolumns
\appendix
%Appendix A
\section{Appendix: Complete Context-Sensitive Algorithm}\label{app:algo}

In this section, algorithm~\ref{fig:csInterFlow} shows the full
pseudo-algorithm for the context-sensitive analysis described
in Section~\ref{sec:analysis}.
Only the handling of the \callt statement is specific to
the context-sensitive algorithm. All other flow functions
are common to the intraprocedural and context-insensitive
analysis, both described in Section~\ref{sec:analysis} as well.

%%%%%%%%%%%%%%%%%%%%%%%%%%%%%%%%%%%%%%%%%%%%%%%%%%%%%%%%%%%%%%%%%%%%%%%%%%%%%%%%%%%%%%%%%%%%
\begin{algorithm}[!hbtp]
\caption{csInterFlow: Context-Sentive Analysis Algorithm}\label{fig:csInterFlow}
%\SetAlgoNoLine
\SetAlgoLined
\LinesNumbered
\DontPrintSemicolon
\SetKwData{Caller}{caller}
\SetKwData{stmt}{s}
\SetKwData{stmtV}{v}
\SetKwData{stmtA}{a}
\SetKwData{stmtF}{f}
\SetKwData{stmtK}{k}
\SetKwData{stmtR}{r}
\SetKwData{stmtT}{t}
\SetKwData{stmtB}{b}
\SetKwData{stmtQ}{q}
\SetKwData{stmtP}{p}
\SetKwData{tainted}{tainted}
\SetKwData{fZero}{$f_0$}
\SetKwData{False}{False}
\SetKwData{Input}{IN}
\SetKwData{Output}{OUT}
\SetKwData{TmpFlow}{tmpFlow}
\SetKwData{summaryTable}{sumTable}
\SetKwFunction{Map}{map}
\SetKwFunction{LCOPY}{COPY}
\SetKwFunction{LLOAD}{LOAD}
\SetKwFunction{LSTORE}{STORE}
\SetKwFunction{LADDROF}{ADDROF}
\SetKwFunction{LCALL}{CALL}
\SetKwFunction{LSOURCE}{SOURCE}
\SetKwFunction{LSINK}{SINK}
\SetKwFunction{Type}{TypeOf}
\SetKwFunction{TaintUse}{taintUse}
\SetKwFunction{Update}{Update}
\SetKwFunction{InterFlow}{csInterFlow}
\SetKwFunction{Taint}{taint}
\SetKwFunction{Func}{func}
\SetKwFunction{First}{first}
\SetKwFunction{Formal}{formal}
\SetKwFunction{Formals}{formals}
\SetKwFunction{Toplevel}{toplevel}
\SetKwFunction{FlowInsert}{flowInsert}
\SetKwInOut{InData}{input}
\SetKwInOut{OutData}{output}


\InData{$\Caller: \mathit{Proc}, \stmt: \mathit{Inst}$}
\OutData{}
\Switch{$\Type(\stmt)$}{
	\Case{$\LCOPY\ [\stmtP = \stmtQ] $}{
		\If{$\stmtQ \in \Input[\stmt]$}{
			$\FlowInsert(\stmt, \stmtP)$;
		}
	}
	\Case{$\LLOAD\ [p = *q] $}{
		\ForEach{$\stmtA_k \in \pointsto{\overline{\stmt}}{q}$}{
			$\stmtT_k \leftarrow \Toplevel(\stmtA_k)$\;					
			\If{$\stmtT_k \in \Input[\stmt]$}{
				$\FlowInsert(\stmt, \stmtP)$\;
				$break$
			}
		}
	}	
	\Case{$\LSTORE\ [*p = q] $}{
		\If{$\stmtQ \in \Input[\stmt]$}{	
			%\ForEach{$\stmtA_k \in \pointsto{\overline{\stmt}}{p}$}{
				%$\stmtT_k \leftarrow \Toplevel(\stmtA_k)$\;					
				%$\Output[\stmtT_k] \leftarrow \Input[\stmtT_k] \cup \Input[\stmtQ] \cup \set{\stmt}$\;				
				$\FlowInsert(\stmt, \stmtP)$;
			%}
		}	
	}	
%Context-sensitive		
	\Case{$\LCALL\ [\stmtR = \mathit{call}\ \Func(a_0, a_1, ..., a_n)]$}{
		\If{$\Caller \neq \Func$}{
			\If{$0 \in \summaryTable[\Func]$}{
				\ForEach{$\stmtF_k \in \Formals(\Func)$}{
					\If{$\summaryTable[\Func][k] = 0$}{
						%\ForEach{$\stmtB \in \pointsto{\overline{\stmt}}{\stmtA_k}$}{
							%$\Input[\stmtF_k] \leftarrow \Input[\stmtF_k] \cup  \Input[\stmtB]$\;		
							$\FlowInsert(\stmt, \stmtA_k)$;											
						%}
					}
				}											
				$\InterFlow(\Caller, \Func)$\;			
				\ForEach{$\stmtF_k \in \Formals(\Func)$}{
					\If{$\summaryTable[\Func][k] = 0\ \aand\ \Output[\stmtF_k] \neq \Input[\stmtF_k]$}{
						%$\Output[\stmtA_k] \leftarrow \Output[\stmtF_k] \cup \set{\stmt}$;			
						$\summaryTable[\Func][k] = 1$\;						
						$\FlowInsert(\stmt, \stmtA_k)$\;				
					}
				}					
			}
		}
	}		
%	\uCase{$\LADDROF\ [p = \&a]$}{
%	}	
%	\Case{$\LSINK [\stmtR = \mathit{call}\ \Func]$}{
%	}		
}
\end{algorithm}


\end{document}
