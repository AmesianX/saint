\section{Analysis}\label{analysis}

We will implement our analysis as a fixed-point iteration
over the set of program statements. The analysis will be
performed on the interprocedural control flow graph (ICFG) of
the program. Intraprocedural flow edges will be used from
the each function control flow graph. Interprocedural edges,
that arise at method calls, will be used from the call graph
built by LLVM.  

Algorithm~\ref{fig:algoAnalyze} shows the pseudo code of the
analysis algorithm we plan to use\footnote{Taken from
lecture notes of Jonathan Aldrich}.

\begin{algorithm}
\caption{Analyze}\label{fig:algoAnalyze}
\SetAlgoLined
\LinesNumbered
\DontPrintSemicolon
\SetKwData{func}{f}
\SetKwData{Input}{input}
\SetKwData{Output}{output}
\SetKwData{varJ}{j}
\SetKwData{varI}{j}
\SetKwData{worklist}{worklist}
\SetKwFunction{Input}{input}
\SetKwFunction{Ouput}{output}
\SetKwFunction{Flow}{Flow}
\SetKwFunction{Update}{Update}
\SetKwFunction{InitDataFlow}{initDataFlow}
\SetKwFunction{Next}{next}
\SetKwFunction{Succs}{succs}
\SetKwInOut{InData}{input}
\SetKwInOut{OutData}{output}


\InData{$\func: Proc, \Input: D, \InitDataFlow: Inst \rightarrow D$}
\OutData{$\Output: D$}
$ s_0 \leftarrow first(\func)$\;
$\Input[s_0] \leftarrow \InitDataFlow(s_0)$\;
$\worklist \leftarrow \set{s_0}$\;
\While{ $\worklist\ !=\ \emptyset$ }{
	$\varI \leftarrow \Next(\worklist)$\;
	$\Output[i] = \Flow(\varI, \Input[\varI])$\;
	\ForEach{ $\varJ \in \Succs(\varJ)$ }{
		\If{ $\Output[\varI] \not\sqsubseteq \Input\varJ]$}{
			$\Input[\varJ] = \Input[\varJ] \sqcup \Output[\varI]$\;
			$\worklist = \worklist \cup\ \set{\varJ}$\;
		}	
	}
}
\end{algorithm}
\DecMargin{1em}

D: Analysis dataflow set\\
Inst: Set of instructions in the program.\\
input: Inst $\rightarrow$ D. Dataflow values at the
program point before an instruction.\\
output: Inst $\rightarrow$ D. Dataflow values at the
program point after an instruction.\\
worklist: Set of instructions to be processed.\\
flow: Inst $\rightarrow$ D. Analysis Dataflow function.\\
next: worklist $\rightarrow$ Inst. Function that returns the
next instructions from the worklist.\\

For each program function $f$,
the analysis will generate a summary and use it whenever
a call to $f$ is encountered.
Section~\ref{sec:summaries} handles function summary generation.

Algorithm~\ref{fig:algoFlow} shows the analysis' transfer functions.
For a pointer variable $q$, we define \(p[s](q)\) as the set
of program locations to which q may point to (i.e $q$'s \textit{points-to set})
at the program statement labelled with $s$.
$In_s$ and $Out_s$ represent the dataflow set \textit{before}
and \textit{after} statement $s$ respectively. 
For a tainted variable $t$, $S_t$ is the set of previous
statements which use $t$ as tainted variable.


Our analysis essentially works in two steps. The
first step consists in creating a summary for each
function.

\IncMargin{1em}
\begin{algorithm}
\caption{Flow}\label{fig:algoFlow}
\SetAlgoLined
\LinesNumbered
\DontPrintSemicolon
\SetKwData{Input}{input}
\SetKwData{Output}{output}
\SetKwData{stmt}{inst}
\SetKwData{stmtV}{v}
\SetKwFunction{Map}{map}
\SetKwFunction{LCOPY}{COPY}
\SetKwFunction{LLOAD}{LOAD}
\SetKwFunction{LSTORE}{STORE}
\SetKwFunction{LADDROF}{ADDROF}
\SetKwFunction{LCALL}{CALL}
\SetKwFunction{Type}{Type}
\SetKwFunction{Update}{Update}
\SetKwFunction{Analyze}{Analyze}
%\SetKwFunction{FindCompress}{FindCompress}
\SetKwInOut{InData}{input}
\SetKwInOut{OutData}{output}


\InData{$\stmt: Inst, \Input: D$}
\OutData{$\Output: D$}
\Switch{$\Type(\stmt)$}{
	\uCase{$\LCOPY\ [p = q]$}{
		$\Output \leftarrow \Input \cup \set{(p, \set{\stmt})}$\;
	}
	\uCase{$\LLOAD\ [p = *q]$}{
		\ForEach{$\stmtV \in \pointsto{\stmt}{q}$}{
			$S_{v} \leftarrow \Map(\stmtV, \Input)$\;		
			$\Output \leftarrow \Input \cup \set{(\stmtV, S_v \cup \stmt)}$\;
		}
	}
%	\Case{$\mathbb{ADDROF}\ [p = \&a]$}{
%	}
	\uCase{$\LSTORE\ [*p = q]$}{
		\ForEach{$\stmtV \in \pointsto{\stmt}{p}$}{
			$S_{v} \leftarrow \Map(\stmtV, \Input)$\;
			$\Output \leftarrow \Input \cup \{ (\stmtV, S_{v} \cup \stmt) \}$\;
		}
	}	
	\uCase{$\LCALL\ [call\ f]$}{
		$\Output \leftarrow Analyze(f, \Input)$
	}		
	\Other{
	}
}
\end{algorithm}
\DecMargin{1em}

\subsection{Motivating Example Case Study}\label{sec:sampleSummary}

\subsection{Summary Generation}\label{sec:summaries}

Function summaries will be generated in the reverse
topological order of the callgraph. Each strongly connected
component (SCC) of the callgraph will be handled as one
node. That is, a single summary will be generated for
mutual recursive functions. In our motivating example
from Figure~\ref{fig:sample}, only one summary will
be generated for functions \even{} and \odd{} because they
are mutually recursive.
Figure~\ref{fig:summary} shows the algorithm for function
summary generation.

Given a function $f$, our analysis will generate and use
a summary for $f$. Let $\set{i_0, i_1, ..., i_j}$ be the set
of $f$'s formal parameters. Let assume that all global variables
are accessible in $f$ through a set of local variables 
$G = \set{g_0, g_1, ..., g_k}$. $f$'s summary is a set of integers 
$\mathcal{S}(f) = \set{0, 1, ...,m, n, ..., z}$.
We now describe f's summary 
defined as follows: 

Figure~\ref{} illustrates the function summary for \main{}
of Figure~\ref{fig:sample}

\subsection{Handling of complex data structures}

\subsection{Context-Sensitivity} 

We achieve context-sensitivity by analyzing callees
with dataflow set information of their formal parameters
at calling sites.

For instance in Figure~\ref{fig:sample}, the call to \even{} at
line $4$ would take into the account the fact that \texttt{x}
is tainted.
On the other hand, \even{}'s call at line $25$ can not safely
make this assumption.