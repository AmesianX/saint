\section{Related Work}\label{sec:related}

\paragraph{\cite{Shankar:2001:DFS}} presents a tool to statically
detect format string vulnerabilities in C programs. Their analysis
is based on type qualifiers and type inference. Given a C program,
an initial subset of the program is annotated with the type qualifiers
\texttt{tainted} and \texttt{untainted}. The paper defines a set
of inference rules, which has a rule for each construct of
the C programming language. The inference rules are used, together
with the initial program annotations to generate type qualifier
constraints over the program. The analysis then warns the user
about a format string vulnerability whenever the program does not
type checks due to an unsatisfiable constraint. The presented tool
is built on top of the type qualifier framework \textit{CQUAL} 
\cite{Foster:pldi99}.

\paragraph{\cite{Dimitru:2009:STAC}} STAC presents a static taint
analysis for C, implemented as a type system. The paper defines a type
domain consisting of the values \texttt{Tainted} and \texttt{Untainted}.
A type environment is a mapping from variables to type domain values.
STAC defines a set of inference rules that allows to compute
type environments for C program statements and functions. The paper
also proves the soundness of the presented taint analysis.

%Dynamic taint analysis

%Taint Analysis for managed languages