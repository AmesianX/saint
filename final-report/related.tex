\section{Related Work}\label{sec:related}

\paragraph{\cite{Shankar:2001:DFS}} presents a tool to statically
detect format string vulnerabilities in C programs. The analysis
is based on \textit{type qualifiers} and type inference. Given a C program,
an initial subset of the program is annotated with the type qualifiers
\texttt{tainted} and \texttt{untainted}, which are new qualifiers
introduced by the authors. The paper defines a set
of inference rules for C language elements. The inference rules are used,
together with the initial program annotation to generate type qualifier
constraints over the program. The analysis then warns the user
about a format string vulnerability whenever the program does not
type checks due to an unsatisfiable constraint. The presented tool
is built on top of the type qualifier framework \textit{CQUAL} 
\cite{Foster:pldi99}.

\paragraph{\cite{Dimitru:2009:STAC}} presents STAC: a static taint
analysis for C, implemented as a type system. The paper defines a type
domain consisting of the values \texttt{Tainted} and \texttt{Untainted}.
A type environment is a mapping from variables to type domain values.
STAC defines a set of inference rules that allows to compute
type environments for C program statements and functions. The paper
also proves the soundness of the presented taint analysis.

\paragraph{\cite{Jovanovic:2006:Pixy}} presents \texttt{Pixy}: a tool
that statically scans for cross site scripting vulnerabilities in PHP
scripts. Pixy and and our project have similar goals. The main
difference is that Pixy performs a static analysis whereas we plan
to develop a dynamic analysis. Pixy implements a flow-sensitive,
interprocedural and context-sensitive dataflow analysis, which
is based on an alias and literal analysis. After evaluating
their tool on three PHP applications, they authors observed 
a false positives rate of around $50\%$.

%Dynamic taint analysis
\paragraph{\cite{Clause:2007:Dytan}} presents \texttt{Dytan}: a generic
framework for implementing dynamic taint analyses. Dytan generates
dynamic analyses that perform on x86 binaries. The source code of the
analyzed application does not need to be available.
Developers specify a dynamic taint analysis with Dytan by giving
(1) taint sources, (2) taint sinks, (3) and a propagation policy.
Taint sources can be variable names, function-return values, and
data read from I/O stream such as a file or a network connection.
Taint sinks are specified by memory/code location or based on the
usage scenario. Memory and code location specifications may include 
a variable's name and scope, a procedure name and the index of a formal parameter, 
a position in the source code, an offset from the base address of the program,
and the entry or exit points of a procedure.
A usage scenario specification takes into account whether the developer
wants check a taint information before the execution instructions of a given type
(e.g., a jump instruction). For that, the developer needs to
specify an instruction type and its corresponding checking operation.
Dytan provides dataflow and control-flow based propagation policies.
Dataflow propagation policies are based on (explicit) data dependencies.
Control-flow dependencies relies on (implicit) control dependencies. 

%Taint Analysis for managed languages