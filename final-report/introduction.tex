\section{Introduction}
Software vulnerabilities are security threats
that exist in an application when users may exercise
unauthorized  control of the application through supplied input.
Such unauthorized control are classified in different
categories depending on how the user gains control of
the application: buffer overflow, format string attacks,
SQL injection, cross site scripting, etc.
Researchers have worked both on dynamic \cite{Clause:2007:Dytan,
Kiezun:2009:Ardilla} and static techniques 
\cite{Jovanovic:2006:Pixy, Shankar:2001:DFS, livshits05finding, 
Avots:2005:ISS, Dimitru:2009:STAC, Tripp:2009:TET} to find 
security vulnerabilities in software. 

This paper presents a static taint analysis to detect software
vulnerabilities in C programs. The analysis is implemented
using the LLVM framework\cite{Lattner:2004:LLVM}. Our analysis
is interprocedural, and developers can choose between a
context-insensitive and a context-sensitive version. 

Taint analysis searches program external inputs usage that
may allow users to gain unauthorized control over the program. 
In taint analysis, a \textit{source} is a program location
or statement that may produce a \textit{tainted value}, i.e,
a value from an untrusted program input (e.g. return value from a
system call, user input). 
A \textit{sink} is program location or statement that may use
a tainted value (e.g. a function call may have a tainted value as
actual argument), and thereby granting unauthorized control
to a malevolent user.
A \textit{sanitizer} is a function that validates tainted
values and make them safe (trusted, untainted) for further
program usage.

Taint analysis proceeds by first tagging values from sources
as tainted. Once tagged, the tainted values are propagated
through the program.
\textit{Taint propagation} is the process of marking as tainted, 
values that result from operations involving tainted data. This
can be an arithmetic operation, an assignment, etc. In the
last step, taint analysis emits a warning whenever a tainted
value is used at a sink location.

Taint propagation can be \textit{data flow} or \textit{control flow}
based. Data flow based taint propagation is explicit and exists
due to data dependencies in the program (e.g. assigning the value
of tainted variable $s_u$ to variable $s_d$).
Control flow based taint propagation is implicit and is due to
control dependencies (e.g. if tainted variable $s_t$ is used in a branch
condition, variables assigned in the condition statement branches
become tainted because their values have a dependency on the
tainted variable $s_t$).

This paper is organized as follows: Section~\ref{sec:example}
introduces our running example. Section~\ref{sec:llvm} gives
a brief overview of the LLVM framework's intermediate
representation. We present our analysis and an experimental
evaluation in Section~\ref{sec:analysis} and Section~\ref{sec:evaluation}
respectively. Finally, we discuss related work in Section~\ref{sec:related}
and conclude in Section~\ref{sec:conclusion}.