\section{Motivating Example}\label{sec:example} 

\begin{figure*}[!ht]
\centering
\EmbedCode{c}{main.c}{left}
\caption{Motivating Example}
\label{fig:sample}
\end{figure*}

Figure~\ref{fig:sample} shows $4$ C functions:
\main, \even, \odd, and \compute.

In \main{}, function \texttt{scanf} from the standard input/ouput
library gets an integer input from user at line $3$
and stores it in variable \texttt{x}. \texttt{x} 
becomes \textit{tainted} since it holds a value from
the environment which has not been validated and
sanitized.
\texttt{x} is later as argument to \even{} and \odd{} at
lines $4$ and $5$ respectively.

In \compute{}, variable \texttt{sum} gets tainted at
line $13$ through function \texttt{scanf}. Parameter
\texttt{x} is tainted only if it was passed tainted
at calling sites.
This is for instance the case in \main{} at line $6$.
Observe that tainted parameter \texttt{x} is used in
the conditional expression at line $12$, which constitute
a case of \textit{control-flow based taint propagation}.
Our analysis will not handle control-flow based taint
propagation\footnote{Detection of some of denial-of-service
vulnerabilities requires handling of control-flow based
taint propagation \cite{Chang:2009:ICS}}. 

Functions \even{} and \odd{} have one formal parameter
\texttt{x}. Both do not taint \texttt{x} in their code.
Thus \texttt{x} may be tainted in their body iff it was
passed as tainted at calling sites. This is the case
for \even{} at line $4$ in \main{}. Thus implying a tainted
parameter for \odd{} at line $32$ in \even{}.