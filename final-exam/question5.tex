\section*{Question 5}

\subsection*{Introduction}
This paper presents an intraprocedural analysis that detects
array accesses for which index variable checking was not done
on some paths leading to the access.

Statically verifying that programs check array index variables
before array accesses is useful at it may help developers detect
incorrect array accesses that may lead to program termination
at runtime.

Our analysis is implemented in the
Clang framework\footnote{http://clang-analyzer.llvm.org/},
which is a static source code analysis tool to find bugs in
C, C++, and objective C.

\subsection*{Intraprocedural Analysis}

The Clang analyzer performs a symbolic execution on the CFG
of the program under analysis. The analysis is path- and
context-sensitive. To comply to the clang framework, we need
to implement our analysis as a \textit{checker}. 

We need to create a \texttt{struct} that will map the status
of array variables to their status.
The state of each array will contain:
\begin{itemize}
\item the index variable associated with it (we will not handle aliasing)
\item a boolean variable that will be set to true once the
index variable has been checked in a conditional statement.
\end{itemize}

\subsection*{Clang Infrastructure Integration}

We implemented our analysis using LLVM 3.3.
\texttt{LLVMSRC} denotes the root folder of the LLVM source tree.

We added our source code file \texttt{Q5ArrayChaker.cpp} in the
clang folder \texttt{LLVMSRC/tools/clang/lib/StaticAnalyzer/Checkers},
and a description of our checker in the file
\texttt{LLVMSRC/tools/clang/lib/StaticAnalyzer/Checkers/Checkers.td}
within the section \textit{Security Checkers}. 
We added our checker in the \textit{alpha package} since it is a new checker.

Verification that the new checker was successfully added occurs with the
command
\begin{verbatim}
clang -cc1 -analyzer-checker-help
\end{verbatim}

Issuing the above command shows our checker as: \textit{alpha.security.Q5ArracyChecker}.






