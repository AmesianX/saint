\section*{Question 3: Paper Summary of "Automatic Testing of Sequential and Concurrent Substitutability"}

\newcommand{\mysub}{substitutability}
\newcommand{\mySub}{Substitutability}

\paragraph{Key Ideas.}
The paper addresses the problem of \textit{\mysub} in 
object-oriented programming languages. \textit{\mySub} is relevant
for languages featuring \textit{polymorphism} and \textit{inheritance}.
The \mysub{} principle stipulates that a subclass of an upper
class shall be safely used whenever it is accessed from a
pointer that has the type of the upper class. That is, it is
semantically safe to use a subclass instance everywhere an
upper class instance could be used.

The authors present a testing technique that helps checking
that code preserves the \mysub{} principle. Their technique
can be used for both sequential and concurrent programs.
Given a class \textit{Super} and its subclass \textit{Sub}, the
presented analysis generates test cases that can test both
\textit{Super} and \textit{Sub} instances. Then, the analysis
uses \textit{Super}'s behavior as an oracle for \textit{Sub}.
The analysis reports a warning whenever \textit{Sub} behaves
differently from \textit{Super}. The analysis comes in two
flavors: it can report any observed divergence between \textit{Super}
and \textit{Sub} or can report only cases when the divergence 
leads to a crash of the application.

The generated tests consist in a sequence of method calls,
specified by their name, their input parameter, and their
output variable if any. The tests are randomly created by
a test generator that can produce both sequential and
concurrent tests. The authors use the Java Pathfinder
model checker to explore all interleavings when running
tests to find concurrent 

\paragraph{Justifications.}


\paragraph{Our opinion.}
The respect of \mysub{} is important for building correct
object-oriented systems. The tool presented by the paper
is therefore very useful to check semantic consistency of
object-oriented software.

The handling of concurrency by the paper is very relevant and
of great importance given the current ubiquitousness of
multi-core architectures.

\paragraph{Discussion Topics.}
\begin{itemize}
\item Since \dl{} adds more code (e.g., stubs) to an existing application, users
may eventually end downloading more data, thus increasing network usage and
costs. For what type of applications would \dl{} still be profitable to 
users ?

\end{itemize}
