\section*{Question 3: Paper Summary of \cite{Pradel:2013:ATS}}

\newcommand{\mysub}{substitutability}
\newcommand{\mySub}{Substitutability}

\paragraph{Key Ideas.}
The paper addresses the problem of \textit{\mysub} in 
object-oriented programming languages. \textit{\mySub} is relevant
for languages featuring \textit{polymorphism} and \textit{inheritance}.
The \mysub{} principle stipulates that a subclass of an upper
class shall be safely used whenever it is accessed from a
pointer that has the type of the upper class. That is, it is
semantically safe to use a subclass instance everywhere an
upper class instance could be used.

The authors present a testing technique that helps checking
that code preserves the \mysub{} principle. Their technique
can be used for both sequential and concurrent programs.
Given a class \textit{Super} and its subclass \textit{Sub}, the
presented analysis generates test cases that can test both
\textit{Super} and \textit{Sub} instances. Then, the analysis
uses \textit{Super}'s behavior as an oracle for \textit{Sub}.
The analysis reports a warning whenever \textit{Sub} behaves
differently from \textit{Super}. The analysis comes in two
flavors: it can report any observed divergence between \textit{Super}
and \textit{Sub} or can report only cases when the divergence 
leads to a crash of the application.

The generated tests consist in a sequence of method calls,
specified by their name, their input parameter, and their
output variable if any. The tests are randomly created by
a test generator that can produce both sequential and
concurrent tests. The authors use the Java Pathfinder
model checker to explore all interleavings when running
tests to find unsafe \mysub{} in concurrent programs.

\paragraph{Two Strengths of the paper.}
\begin{enumerate}
\item The paper addresses the important problem of \mysub{} and
gives specific and precise tools to identify \mysub{} problems
in real code. Their technique also finds \mysub{} problems
that might occur during concurrent execution.

\item The paper provides a specific method to distinguish
between \textit{crashing substitute} and \textit{output diverging
substitute}. This might reduce false positives, and help developers
under time pressure to focus on more severe problems.
\end{enumerate}

\paragraph{Two Weaknesses of the paper.}
\begin{enumerate}
\item The technique misses cases where undesired behaviors
might come from the upper class because it uses the super class'
behavior as test oracle.

For instance in Question 2, the upper class \texttt{A} throws
an exception when a call to method \texttt{add} is made. On the
other hand, the behavior of \texttt{A}'s subclass \texttt{B}
is correct since it does not throw an exception after a call
to \texttt{add} (adding an element to the array \texttt{data}).

\item The paper does not have an example showing a test generation,
including input parameters of their method. 
it does not also show an example where class fields are the source
of a divergent behavior.
This might make it difficult for the reader to evaluate the correctness
of their approach just by reading the paper. The reader might have
to manually generates the tests, and run the
\textit{Superclass oracle} himself.

\end{enumerate}

\paragraph{Our opinion.}
The respect of \mysub{} is important for building correct
object-oriented systems. The technique presented by the paper
is therefore very useful to check semantic consistency of
object-oriented software.

Also, the handling of concurrency is very relevant and of great
importance given the current ubiquitousness of multi-core
architectures.

We would recommend this paper for publication.