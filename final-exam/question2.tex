\section*{Question 2: Application of Techniques from \cite{Pradel:2013:ATS}}

\newcommand{\myA}{\texttt{A}}
\newcommand{\myB}{\texttt{B}}
\newcommand{\myout}{OUT}

Given class \myA{} and its subclass \myB{} from the exam
sheet, we are going to show that class \myB{} is a
\textit{crashing substitute} of its super class \myA{}.

\paragraph*{Constructor Mapping}
Given the fact that the subclass (\myB{}) and the super
class (\myA{}) do not have the exact same types of arguments,
we define the following constructor mapping that show the
equivalence between constructor calls of \myA{} and \myB{}:\

$\mathit{B}(int) \rightarrow \mathit{A}()$

\paragraph*{Test Generation}
We are going to focus on sequential tests, even though the technique
described in \cite{Pradel:2013:ATS} might also generate concurrent
tests.
The technique described in \cite{Pradel:2013:ATS} might generate
the tests in Figure~\ref{fig:test} and Figure~\ref{fig:divergence}.

\paragraph*{Output Oracle}
\begin{figure}[!ht]
\EmbedCode{java}{Test.java}{left}
\caption{Sample Generated Test that do not discover divergence}
\label{fig:test}
\end{figure}

Figure~\ref{fig:test} illustrates a test that do not exhibit
any visible difference between the super class \myA{} and its subclass
\myB{}. The first part of the tests creates the
\textit{objects under test} (\myout{}) \texttt{sup} and \texttt{sub}
in lines $4$ and $9$ respectively.
The second part of the test calls method \texttt{size()} on both instances
(lines $5$ and $10$ respectively).

According to classes \myA{} and \myB{} code, the output of
both test methods \texttt{test2SUP} and \texttt{test2SUB} will
be \textbf{int:0} and \textbf{int:1} respectively.

The output \textbf{int:0} from \texttt{test2SUP} will then be used
by the \textit{Superclass oracle} to evaluate the result \textbf{int:1}
of \texttt{test2SUB}.
In this case, the \textit{output oracle} will analyze both output and will
not find any divergence in their visible behavior since both tests
generated an integer.

\paragraph*{Crashing Oracle}
\begin{figure}[!ht]
\EmbedCode{java}{Divergence.java}{left}
\caption{Sample Generated Test that exhibits \myB{} as a crashing substitute}
\label{fig:divergence}
\end{figure}

Figure~\ref{fig:divergence} shows a test that exhibits \myB{}
as an unsafe substitute for its super class \myA{}.
The first part of the tests creates the (\myout{}) \texttt{sup}
and \texttt{sub} in lines $4$ and $9$ respectively.
The second part of the test calls method \texttt{size()} on
both instances (line $5$ and $10$ respectively).

According to class \myA{} code, method \texttt{test1SUP} will
generate \texttt{UnsupportedOperationException} exception while
\texttt{test1SUB} will terminate with no failure or exception.

Since the behavior of class \myA{} (\texttt{test1SUP}) will be
used as test oracle for \texttt{test1SUB}, the crashing oracle will
find a divergence between class \myA{} and its subclass \myB{}.